\documentclass[pscyr,specification,annotation]{itmo-student-thesis}

%% Опции пакета:
%% - specification - если есть, генерируется задание, иначе не генерируется
%% - annotation - если есть, генерируется аннотация, иначе не генерируется
%% - times - делает все шрифтом Times New Roman, собирается с помощью xelatex
%% - pscyr - делает все шрифтом Times New Roman, требует пакета pscyr.

%% Делает запятую в формулах более интеллектуальной, например:
%% $1,5x$ будет читаться как полтора икса, а не один запятая пять иксов.
%% Однако если написать $1, 5x$, то все будет как прежде.
\usepackage{icomma}

%% Один из пакетов, позволяющий делать таблицы на всю ширину текста.
\usepackage{tabularx}

%% Данные пакеты необязательны к использованию в бакалаврских/магистерских
%% Они нужны для иллюстративных целей
%% Начало
\usepackage{tikz}
\usetikzlibrary{arrows}
\usepackage{filecontents}

\begin{document}

\studygroup{M3439}
\title{Оптимизация функции, задаваемой регрессионным лесом}
\author{Ягламунов Владислав Радикович}{Ягламунов В.Р.}
\supervisor{Фильченков Андрей Александрович}{Фильченков А.А.}{доцент, к.ф.-м.н.}{}
\publishyear{2019}
%% Дата выдачи задания. Можно не указывать, тогда надо будет заполнить от руки.
% \startdate{01}{сентября}{2018}
%% Срок сдачи студентом работы. Можно не указывать, тогда надо будет заполнить от руки.
% \finishdate{31}{мая}{2019}
%% Дата защиты. Можно не указывать, тогда надо будет заполнить от руки.
% \defencedate{15}{июня}{2019}

% \addconsultant{Белашенков Н.Р.}{канд. физ.-мат. наук, без звания}
% \addconsultant{Беззубик В.В.}{без степени, без звания}

\secretary{Павлова О.Н.}

%% Задание
%%% Техническое задание и исходные данные к работе
\technicalspec{Требуется 
разработать алгоритм поиска областей минимума и максимума в данном обученном случайном регрессионном лесе.
Требуется минимизировать время работы алгоритма. Алгоритм должен возвращать точный ответ или ответ отличающийся
от точного не более чем не заданную величину.
}

%%% Содержание выпускной квалификационной работы (перечень подлежащих разработке вопросов)
\plannedcontents{Описание существующих решений для оптимизации функции, 
задаваемой регрессионным лесом. Разработка и реализация различных алгоритмов, решающих поставленную задачу.
Сравнение разработанных алгоритмов между собой и существующими решениями задачи.
}

%%% Исходные материалы и пособия 
\plannedsources{}

%%% Цель исследования
\researchaim{Разработка эффективного алгоритма оптимизации функции, заданной регрессионным лесом.}

%%% Задачи, решаемые в ВКР
\researchtargets{\begin{enumerate}
    \item реализация интерфейса для работы с обученным случным регрессионным лесом;
    \item разработка алгоритмов оптимизации функции;
    \item разработка тестирующей системы для алгоритмов оптимизации,
          позволяющей автоматическое тестирование на различных выборках и с набором заданных параметров;
    \item сравнение и анализ работы разработанных алгоритмов, сопоставление с существующими решениями.
\end{enumerate}}

%%% Использование современных пакетов компьютерных программ и технологий
\addadvancedsoftware{Пакет \texttt{scikit-learn} с реализацией современных алгоритмов машинного обучения
                     на языке \texttt{Python}}{****}

%%% Краткая характеристика полученных результатов 
\researchsummary{Получен алгоритм для нахождения оптимума функции, заданной случным лесом,
с возможностью настройки необходимой точности}

%%% Гранты, полученные при выполнении работы 
\researchfunding{При выполнении работы грантов получено не было.}

%%% Наличие публикаций и выступлений на конференциях по теме выпускной работы
\researchpublications{Отсутствуют.}

%% Эта команда генерирует титульный лист и аннотацию.
\maketitle{Бакалавр}

%% Оглавление
\tableofcontents

%% Макрос для введения. Совместим со старым стилевиком.
\prefacepage{}

Существует множество алгоритмов, использующих суррогатные функции для аппроксимации или предсказание различных процессов.
Случайный регрессионный лес часто может применяться в качестве такой функции, так как одно из его положительных качеств ---
возможность эффективно пересчитывать лес при добавлении новой информации. Так на пример, случайный лес может использоваться
в качестве регрессионной модели для реализации алгоритмов последовательной оптимизации основанной на модели 
(Sequential Model-Based Optimization --- SMBO)
Однако, сейчас не существует эффективных оптимизации и применяются неэффективные алгоритмы, как, например, перебор случайных
точек пространства.

%% Начало содержательной части.
\chapter{Описание предметной области и анализ существующих решений}

\section{Случайный регрессионный лес}\label{sec:random_forest}
\subsection{Определение}
Случайный лес (\texttt{Random Forest}) --- алгоритм машинного обучения, основанный на применении ансамбля деревьев принятия решения.
Регрессионный лес используется для решения задач предсказания некоторого численного значения.

\subsection{Обучение}
Для обучения регрессионного леса, исходная выборка разбивается на случайные подвыборки с повторениями. 
После чего, на каждой выборке обучается отдельное дерево принятия решений.
Решение принимается как усреднение (возможно взвешенное) по всем деревьям.


\section{Задача оптимизации}
\subsection{Определение}
Оптимизация --- задача нахождения экстремума (минимума или максимума) целевой функции в некоторой области. 
Так как случайный лес разбивает пространство на конечное число участков, то имеет место случай
комбинаторной оптимизации.

\subsection{Методы}

\section{Постановка задачи}

\section{Анализ существующих решений}

\chapter{Эвристический алгоритм}

\chapter{Алгоритм имитации отжига}

\chapter{Тестирование}

\chapterconclusion{}

В конце каждой главы желательно делать выводы.

%% Макрос для заключения. Совместим со старым стилевиком.
\conclusionpage{}

В данном разделе размещается заключение.

\printmainbibliography{}

\end{document}
