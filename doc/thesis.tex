% \documentclass[pscyr,specification,annotation]{itmo-student-thesis}
\documentclass[pscyr]{itmo-student-thesis}

%% Опции пакета:
%% - specification - если есть, генерируется задание, иначе не генерируется
%% - annotation - если есть, генерируется аннотация, иначе не генерируется
%% - pscyr - делает все шрифтом Times New Roman, требует пакета pscyr.
%% - times - делает все шрифтом Times New Roman, собирается с помощью xelatex

\usepackage{graphicx}
\graphicspath{{logo/}{pics/}}

%% Делает запятую в формулах более интеллектуальной, например:
%% $1,5x$ будет читаться как полтора икса, а не один запятая пять иксов.
%% Однако если написать $1, 5x$, то все будет как прежде.
\usepackage{icomma}

%% Один из пакетов, позволяющий делать таблицы на всю ширину текста.
\usepackage{tabularx}

%% Данные пакеты необязательны к использованию в бакалаврских/магистерских
%% Они нужны для иллюстративных целей
%% Начало
\usepackage[russian]{cleveref}
\usepackage{tikz}
\usepackage{csvsimple}
\usetikzlibrary{arrows}
\usepackage{filecontents}
\usepackage{booktabs}
\addbibresource{thesis.bib}

\usepackage{pgfplots, pgfplotstable}
\usepgfplotslibrary{fillbetween}
\pgfplotsset{compat=newest}

\makeatletter
\pgfplotsset{%
    /pgfplots/flexible xticklabels from table/.code n args={3}{%
        \pgfplotstableread[#3]{#1}\coordinate@table
        \pgfplotstablegetcolumn{#2}\of{\coordinate@table}\to\pgfplots@xticklabels
        \let\pgfplots@xticklabel=\pgfplots@user@ticklabel@list@x
    }
}
\makeatother%
\definecolor{basicno}{rgb}{0.624,0.0,0.0}
\definecolor{basicsome}{rgb}{0.424,0.0,0.0}
\definecolor{random}{rgb}{0.0,0.100,0.424}
\definecolor{gena}{rgb}{0.0,0.100,0.624}
\definecolor{heuno}{rgb}{0.2,0.524,0.2}
\definecolor{heusome}{rgb}{0.2,0.424,0.2}
\definecolor{heumore}{rgb}{0.1,0.324,0.1}

\newcommand{\addplottime} [2] {%
\addplot 
    [color=#2, fill=#2, fill opacity=0.33]
    plot [error bars/.cd, error bar style={line width=1pt}, y dir = both, y explicit]
    table[x expr=\coordindex, y=mean_time, y error=std_time]{#1};
}

\newenvironment{timeaxis} [2] {%
\begin{axis}[
    enlarge x limits=0.2,
    ybar=0pt, 
    width=#1,
    height=#2,
    bar width=8pt,
    ymin=0,
    xlabel=Датасет,
    ylabel=Время работы в сек,
    flexible xticklabels from table={../data/big/heuristic-15.csv}{algo}{col sep=comma},
    xticklabel style={text height=1.5ex, font=\tiny}, % To make sure the text labels are nicely aligned
    xtick=data,
    % legend style={at={(0.02,0.98)},anchor=north west},
]
}{%
\end{axis}
}

\newcommand{\timebig} [2] {%
\pgfplotstableread[col sep=comma]{../data/big/basic-00.csv}\basicno
\pgfplotstableread[col sep=comma]{../data/big/basic-05.csv}\basicsome
\pgfplotstableread[col sep=comma]{../data/big/heuristic-00.csv}\heuno
\pgfplotstableread[col sep=comma]{../data/big/heuristic-05.csv}\heusome
\pgfplotstableread[col sep=comma]{../data/big/heuristic-15.csv}\heumore
\pgfplotstableread[col sep=comma]{../data/big/random.csv}\random

\begin{tikzpicture}
\begin{timeaxis}{#1}{#2}

\addplottime{\basicno}{basicno}
\addplottime{\basicsome}{basicsome}
\addplottime{\random}{random}
\addplottime{\heuno}{heuno}
\addplottime{\heusome}{heusome}
\addplottime{\heumore}{heumore}

\legend{%
    Перебор, 
    Перебор с $<5\%$, 
    Случайный, 
    Эвристика, 
    Эвристика с $<5\%$, 
    Эвристика с $<15\%$ 
}

\end{timeaxis}
\end{tikzpicture}
}

\newcommand{\timeeasy} [2] {%
\pgfplotstableread[col sep=comma]{../data/easy/basic-00.csv}\basicno
\pgfplotstableread[col sep=comma]{../data/easy/basic-05.csv}\basicsome
\pgfplotstableread[col sep=comma]{../data/easy/heuristic-00.csv}\heuno
\pgfplotstableread[col sep=comma]{../data/easy/heuristic-05.csv}\heusome
\pgfplotstableread[col sep=comma]{../data/easy/heuristic-15.csv}\heumore
\pgfplotstableread[col sep=comma]{../data/easy/random.csv}\random

\begin{tikzpicture}
\begin{timeaxis}{#1}{#2}

\addplottime{\basicno}{basicno}
\addplottime{\basicsome}{basicsome}
\addplottime{\random}{random}
\addplottime{\heuno}{heuno}
\addplottime{\heusome}{heusome}
\addplottime{\heumore}{heumore}

\legend{%
    Перебор, 
    Перебор с $<5\%$, 
    Случайный, 
    Эвристика, 
    Эвристика с $<5\%$, 
    Эвристика с $<15\%$ 
}

\end{timeaxis}
\end{tikzpicture}
}

\newcommand{\timetrees} [2] {%
\pgfplotstableread[col sep=comma]{../data/trees/basic-00.csv}\basicno
\pgfplotstableread[col sep=comma]{../data/trees/basic-05.csv}\basicsome
\pgfplotstableread[col sep=comma]{../data/trees/heuristic-00.csv}\heuno
\pgfplotstableread[col sep=comma]{../data/trees/heuristic-05.csv}\heusome
\pgfplotstableread[col sep=comma]{../data/trees/heuristic-15.csv}\heumore
\pgfplotstableread[col sep=comma]{../data/trees/random.csv}\random

\begin{tikzpicture}
\begin{timeaxis}{#1}{#2}

% \addplottime{\basicno}{basicno}
% \addplottime{\basicsome}{basicsome}
\addplottime{\random}{random}
\addplottime{\heuno}{heuno}
\addplottime{\heusome}{heusome}
\addplottime{\heumore}{heumore}

\legend{%
    % Перебор, 
    % Перебор с $<5\%$, 
    Случайный, 
    Эвристика, 
    Эвристика с $<5\%$, 
    Эвристика с $<15\%$ 
}

\end{timeaxis}
\end{tikzpicture}
}

\newcommand{\addploterror}[2] {% 
\addplot 
    [color=#2, fill=#2, fill opacity=0.33]
    plot [error bars/.cd, error bar style={line width=1pt}, y dir = both, y explicit] 
    table[x expr=\coordindex, y=mean_error, y error=std_error]{#1}; 
}

\newenvironment{erroraxis}[3] {% 
\begin{axis}[
    ytick={0,0.1,...,0.9},
    enlarge x limits=0.2,
    ybar=0pt, 
    width=#1,
    height=#2,
    bar width=8pt,
    ymin=0,
    xlabel=Набор данных,
    ylabel=Отоносительная погрешность,
    flexible xticklabels from table={#3}{algo}{col sep=comma},
    xticklabel style={text height=1.5ex, font=\tiny}, % To make sure the text labels are nicely aligned
    xtick=data,
    legend cell align={left},
    legend style={at={(0.02,0.98)},anchor=north west},
]
}{%
\end{axis}
}

\newcommand{\errorbig}[2] {%
\pgfplotstableread[col sep=comma]{../data/extra/heuristic-05.csv}\heusome
\pgfplotstableread[col sep=comma]{../data/extra/heuristic-15.csv}\heumore
\pgfplotstableread[col sep=comma]{../data/extra/random.csv}\random

\begin{tikzpicture}
\begin{erroraxis}{#1}{#2}{../data/extra/heuristic-15.csv}

\addploterror{\random}{random} 
\addploterror{\heusome}{heusome}
\addploterror{\heumore}{heumore}

\legend{%
    Случайный, 
    Ветви/границы + эвристика с $<5\%$, 
    Ветви/границы + эвристика с $<15\%$ 
}

\end{erroraxis}
\end{tikzpicture} 
}

\newcommand{\erroreasy}[2] {%
\pgfplotstableread[col sep=comma]{../data/easy/basic-05.csv}\basicsome
\pgfplotstableread[col sep=comma]{../data/easy/heuristic-05.csv}\heusome
\pgfplotstableread[col sep=comma]{../data/easy/heuristic-15.csv}\heumore
\pgfplotstableread[col sep=comma]{../data/easy/random.csv}\random
\pgfplotstableread[col sep=comma]{../data/easy/gena.csv}\gena

\begin{tikzpicture} 
\begin{erroraxis}{#1}{#2}{../data/easy/heuristic-15.csv}

\addploterror{\random}{random}
\addploterror{\gena}{gena}
\addploterror{\basicsome}{basicsome}
\addploterror{\heusome}{heusome}
\addploterror{\heusome}{heumore}

\legend{%
    Случайный, 
    Метод имитации отжига, 
    Ветви/границы с $<5\%$, 
    Ветви/границы + эвристика с $<5\%$, 
    Ветви/границы + эвристика с $<15\%$ 
}

\end{erroraxis}
\end{tikzpicture}
}

\newcommand{\errortimeaddplot}[2] {%
\addplot 
    [scatter, color=#2, style={ultra thick}] 
    table[x=algo, y=mean_time]{#1};
\addplot 
    [name path=upper, color=#2] 
    table[x=algo, y expr=\thisrow{mean_time}-\thisrow{std_time}]{#1};
\addplot 
    [name path=lower, color=#2] 
    table[x=algo, y expr=\thisrow{mean_time}+\thisrow{std_time}]{#1};
\addplot 
    [fill, fill opacity=0.2, color=#2] 
    fill between[of=upper and lower];
}

\newcommand{\errortotime}[2] {%
\pgfplotstableread[col sep=comma]{../data/datasets/time_to_error/diabetes70-None.csv}\diabet
\pgfplotstableread[col sep=comma]{../data/datasets/time_to_error/kin8nm30-None.csv}\kinmerrtime
\pgfplotstableread[col sep=comma]{../data/datasets/time_to_error/house_8L37-None.csv}\house
\pgfplotstableread[col sep=comma]{../data/datasets/time_to_error/house_16H25-None.csv}\housebig

\begin{tikzpicture}
\begin{axis}[
    xtick={0.00,0.05,0.10,0.15,0.20,0.25},
    x tick label style={/pgf/number format/.cd,
            fixed, fixed zerofill, precision=2, /tikz/.cd},
    enlarge x limits=0.05,
    width=#1,
    height=#2,
    ymin=0,
    xlabel=Заданная допутсимая погрешность алгоритма,
    ylabel=Время работы в сек.,
    scatter/classes={a={mark=o}},
]

\errortimeaddplot{\diabet}{violet}
\errortimeaddplot{\kinmerrtime}{random}
\errortimeaddplot{\house}{basicno}
\errortimeaddplot{\housebig}{heuno}

\legend{%
    diabetes-70 деревьев,,,,
    kin8nm-30 деревьев,,,,
    house\_8L-37 деревьев,,,,
    house\_16H-25 деревьев,,
}

\end{axis}
\end{tikzpicture}
}

\newcommand{\addplotsmac}[5] {%
\addplot 
    [scatter, color=#2]
    plot [error bars/.cd, error bar style={line width=1pt}, y dir = both,
    y explicit] table[x index=#3, y index=#4, y error index=#5]{#1}; 
}

\newcommand{\smacsize} [2] {%
\pgfplotstableread[col sep=comma, header=false]{../data/smac/limits/new_iris_forest.csv}\forestiris
\pgfplotstableread[col sep=comma, header=false]{../data/smac/limits/new_iris_random.csv}\randomiris
\pgfplotstableread[col sep=comma, header=false]{../data/smac/limits/new_leter_forest.csv}\forestletter
\pgfplotstableread[col sep=comma, header=false]{../data/smac/limits/new_leter_random.csv}\randomletter
\pgfplotstableread[col sep=comma, header=false]{../data/smac/limits/new_gina_forest.csv}\forestgina
\pgfplotstableread[col sep=comma, header=false]{../data/smac/limits/new_gina_random.csv}\randomgina

\begin{tikzpicture}
\begin{axis}[
    enlarge x limits=0.2,
    width=#1,
    height=#2,
    ymin=0,
    xlabel=Размер пространства гиперпараметров,
    ylabel=обратная оценка точности,
    xtick=data,
    scatter/classes={a={mark=o}},
    legend style={at={(0.02,0.98)},anchor=north west},
]

\addplotsmac{\forestiris}{heusome}{2}{4}{5}
\addplotsmac{\forestletter}{basicsome}{2}{4}{5}
\addplotsmac{\forestgina}{gena}{2}{4}{5}
\addplotsmac{\randomiris}{heuno}{2}{4}{5}
\addplotsmac{\randomletter}{basicno}{2}{4}{5}
\addplotsmac{\randomgina}{random}{2}{4}{5}

\legend{%
    iris,
    leter,
    gina-agnositc,
}

\end{axis}
\end{tikzpicture}
}

\newcommand{\smaccount} [2] {%
\pgfplotstableread[col sep=comma, header=false]{../data/smac/runcount/iris_forest.csv}\forestiris
\pgfplotstableread[col sep=comma, header=false]{../data/smac/runcount/iris_random.csv}\randomiris
\pgfplotstableread[col sep=comma, header=false]{../data/smac/runcount/leter_forest.csv}\forestletter
\pgfplotstableread[col sep=comma, header=false]{../data/smac/runcount/leter_random.csv}\randomletter
\pgfplotstableread[col sep=comma, header=false]{../data/smac/runcount/gina_forest.csv}\forestgina
\pgfplotstableread[col sep=comma, header=false]{../data/smac/runcount/gina_random.csv}\randomgina

\begin{tikzpicture}
\begin{axis}[
    enlarge x limits=0.2,
    width=#1,
    height=#2,
    ymin=0,
    xlabel=Количество запусков целевого алгоритма,
    ylabel=обратная оценка точности,
    xtick=data,
    scatter/classes={a={mark=o}},
]
\addplotsmac{\forestiris}{heusome}{0}{3}{4}
\addplotsmac{\forestletter}{basicsome}{0}{3}{4}
\addplotsmac{\forestgina}{gena}{0}{3}{4}
\addplotsmac{\randomiris}{heuno}{0}{3}{4}
\addplotsmac{\randomletter}{basicno}{0}{3}{4}
\addplotsmac{\randomgina}{random}{0}{3}{4}

\legend{%
    iris,
    leter,
    gina-agnositc,
}

\end{axis}
\end{tikzpicture}
}

\newcommand{\smacgeneral}{%
\pgfplotstableread[col sep=comma, header=false]{../data/smac/forest.csv}\forest
\pgfplotstableread[col sep=comma, header=false]{../data/smac/random.csv}\random

\pgfplotstablecreatecol[copy column from table={\random}{[index] 2}] {4} {\forest}
\pgfplotstablecreatecol[copy column from table={\random}{[index] 3}] {5} {\forest}

\pgfplotstableset{%
    bold/.style = {%
        postproc cell content/.style={%
            @cell content/.add={\boldmath}{},
        },
    },
}

\pgfplotstabletypeset[
    every head row/.style={%
        output empty row,
        before row={%
            \hline
            \multicolumn{1}{|c}{Алгоритм} & \multicolumn{1}{|c}{Набор данных} &
            \multicolumn{3}{|c}{Модификация} & \multicolumn{3}{|c|}{Оригинал} \\
        },
    },
    before row=\hline,
    every last row/.style={after row=\hline},
    columns={0, 1, forestacc, plusminus, 3, randomacc, plusminus, 5},
    display columns/0/.style= {%
            column name=Алгоритм,
            string type,
            column type = {|c|},
    },
    create on use/plusminus/.style={create col/set={$\pm$}},
    columns/plusminus/.style= {column name=, string type, column type = {b{0mm}}},
    display columns/1/.style= {%
            column name=Выборка,
            string type,
            column type = {c|},
    },
    create on use/forestacc/.style={%
        create col/expr={1-\thisrow{2}}
    },
    columns/forestacc/.style= {%
            column name=,
            fixed,
            precision=2,
            column type = {b{6mm}},
    },
    columns/3/.style= {%
            column name=,
            fixed,
            precision=2,
            column type = {l|},
    },
    create on use/randomacc/.style={%
        create col/expr={1-\thisrow{4}}
    },
    columns/randomacc/.style= {%
            column name=,
            fixed,
            precision=2,
            column type = {b{6mm}},
    },
    columns/5/.style= {%
            column name=,
            fixed,
            precision=2,
            column type = {l|},
    },
    every row 0 column 2/.style = {bold},
    every row 3 column 2/.style = {bold},
    every row 4 column 2/.style = {bold},
    every row 5 column 2/.style = {bold},
    every row 8 column 2/.style = {bold},
    every row 9 column 2/.style = {bold},
    every row 13 column 2/.style = {bold},
    every row 0 column 3/.style = {bold},
    every row 3 column 3/.style = {bold},
    every row 4 column 3/.style = {bold},
    every row 5 column 3/.style = {bold},
    every row 8 column 3/.style = {bold},
    every row 9 column 3/.style = {bold},
    every row 13 column 3/.style = {bold},
    every row 0 column 4/.style = {bold},
    every row 3 column 4/.style = {bold},
    every row 4 column 4/.style = {bold},
    every row 5 column 4/.style = {bold},
    every row 8 column 4/.style = {bold},
    every row 9 column 4/.style = {bold},
    every row 13 column 4/.style = {bold},
]{\forest}
}


\begin{document}
  
\studygroup{M3439}
\title{Оптимизация функции, задаваемой регрессионным лесом}
\author{Ягламунов Владислав Радикович}{Ягламунов В.Р.}
\supervisor{Фильченков Андрей Александрович}{Фильченков А.А.}{доцент, к.ф.-м.н.}{}
\publishyear{2019}
%% Дата выдачи задания. Можно не указывать, тогда надо будет заполнить от руки.
% \startdate{01}{сентября}{2018}
%% Срок сдачи студентом работы. Можно не указывать, тогда надо будет заполнить от руки.
% \finishdate{31}{мая}{2019}
%% Дата защиты. Можно не указывать, тогда надо будет заполнить от руки.
% \defencedate{15}{июня}{2019}

% \addconsultant{Белашенков Н.Р.}{канд. физ.-мат. наук, без звания}
% \addconsultant{Беззубик В.В.}{без степени, без звания}

\secretary{Павлова О.Н.}

%% Задание
%%% Техническое задание и исходные данные к работе
\technicalspec{Требуется разработать алгоритм поиска областей минимума
и максимума в данном обученном случайном регрессионном лесе. Требуется
минимизировать время работы алгоритма. Алгоритм должен возвращать точный ответ
или ответ отличающийся от точного не более чем не заданную величину. }

%%% Содержание выпускной квалификационной работы (перечень подлежащих разработке вопросов)
\plannedcontents{Описание существующих решений для оптимизации функции,
задаваемой регрессионным лесом. Разработка и реализация различных алгоритмов,
решающих поставленную задачу. Сравнение разработанных алгоритмов между собой
и существующими решениями задачи. }

%%% Исходные материалы и пособия
\plannedsources{}

%%% Цель исследования
\researchaim{Разработка эффективного алгоритма оптимизации функции, заданной регрессионным лесом.}

%%% Задачи, решаемые в ВКР
\researchtargets{\begin{enumerate}
    \item реализация интерфейса для работы с обученным случным регрессионным лесом;
    \item разработка алгоритмов оптимизации функции;
    \item интеграция разработанного алгоритма оптимизации случайного леса в существующие алгоритмы
    \item разработка тестирующей системы для алгоритмов оптимизации, позволяющей автоматическое
    тестирование на различных выборках и с набором заданных параметров;
    \item сравнение и анализ работы разработанных алгоритмов, сопоставление
    с существующими решениями.
    \end{enumerate}}

%%% Использование современных пакетов компьютерных программ и технологий
\addadvancedsoftware{%
    Язык программирования \texttt{C++}, для реализации алгоритма оптимизации
}{**todo: ref**}
\addadvancedsoftware{%
    Язык программирования \texttt{Python}, для тестирования и работы с машинным обучением
}{**todo: ref**}
\addadvancedsoftware{%
    Пакет \texttt{scikit-learn} с реализацией современных алгоритмов машинного обучения на языке \texttt{Python}
}{**todo: ref**}
\addadvancedsoftware{%
    Пакеты \texttt{automl: SMAC и random\_forest\_run}, для сравнительного анализа
}{**todo: ref**}

%%% Краткая характеристика полученных результатов
\researchsummary{Получен алгоритм для нахождения оптимума функции, заданной
случным регрессионным лесом, с возможностью настройки необходимой точности,
а также ограничения области поиска.}

%%% Гранты, полученные при выполнении работы
\researchfunding{При выполнении работы грантов получено не было.}

%%% Наличие публикаций и выступлений на конференциях по теме выпускной работы
\researchpublications{Отсутствуют.}

%% Эта команда генерирует титульный лист и аннотацию.
\maketitle{Бакалавр}

%% Оглавление
\tableofcontents

%% Макрос для введения. Совместим со старым стилевиком.
\startprefacepage

Существует множество алгоритмов, использующих суррогатные функции для
аппроксимации или предсказания различных процессов. Случайный регрессионный лес
часто может применяться в качестве такой функции, так как одно из его
положительных качеств --- возможность эффективно пересчитывать лес при
добавлении новой информации. Так на пример, случайный лес может использоваться
в качестве регрессионной модели для реализации алгоритмов последовательной
оптимизации основанной на модели (Sequential Model-Based Optimization ---
SMBO\cite{smac}) Однако, функция заданная таким образом является трудно обратимой
и сейчас не существует эффективных алгоритмов оптимизации и на практике
применяются не оптимальные алгоритмы, как, например, перебор случайных точек
пространства или локальный поиск.

%% Начало содержательной части.

\chapter{Описание предметной области и анализ существующих решений}

\section{Дерево принятия решений}

Дерево принятия решений (\texttt{Decision Tree}) --- модель применяющаяся
в алгоритмах машинного обучения, цель которой состоит в том, чтобы предсказать
значение целевой функции в заданной точке. Структура представляет собой
подвешенное дерево (часто двоичное), где в каждой вершине происходит разбиение
пространства по одному и признаков. Тем самым спускаясь по дереву до листа,
находится в какой области лежит заданная точка (лист дерева соответствующий
данной области), после чего возвращается значение функции в полученном листе
дерева.

Существуют различные алгоритмы для построения дерева, такие как: ID3, C4.5, CART
и другие. Обычно применяется построение сверху вниз, где на каждом шаге
происходит разбиение пространства по некоторому признаку, таким образом чтобы
максимизировать значение выбранной метрики. Часто применяющиеся метрики включают
себя: критерий Джини, информационный выигрыш и понижение дисперсии. Для избежания
переобучения, в конце работы алгоритма используется отсечение ветвей.

Преимуществами деревьев принятия решения по сравнению с другими моделями
машинного обучения являются:

\begin{enumerate}
    \item Понятная человеку интерпретация процесса принятия решения.
    \item Независимость от характера признаков --- может одинаково работать как
    целочисленными и вещественными признаками, так и с категориальными.
    \item Эффективная скорость обучения, даже при большом наборе данных.
\end{enumerate}

\section{Случайный регрессионный лес}

\begin{figure}[ht!]
\caption{Разбиение пространства случайным лесом}\label{random_forest}
\includegraphics[height=0.35\textheight]{random_forest.png}
\end{figure}

Случайный лес (\texttt{Random Forest}\cite{randomforest}) --- модель машинного
обучения, основанная на применении ансамбля деревьев принятия решения, где
каждое дерево обучается независимо от остальных. Итоговый результат получается
как среднее (взвешенное среднее) по все ответам отдельных деревьев. Так как
каждое дерево разбивает пространство на прямоугольные области, где оно
возвращает одинаковые значения, то итоговый лес является пересечением всех таких
разбиений (Рисунок~\ref{random_forest}). Случайный лес называется регрессионным,
если он используется для решения задач предсказания некоторого численного
значения.

Для обучения регрессионного леса, исходные входные данные разбиваются на случайные
подвыборки с повторениями. После чего, на каждой выборке обучается отдельное
дерево принятия решений. Так же применяется метод \texttt{feature bagging}, где
деревья обучаются не на полном наборе признаков, а только на случено выбранном
его подмножестве.

\section{Постановка задачи}

В данной работе рассматривается задача оптимизации случайного регрессионного
леса.

Оптимизация --- задача нахождения экстремума (минимума или максимума) целевой
функци в некоторой области. Так как случайный лес разбивает пространство на
конечное число участков, то имеет место случай комбинаторной оптимизации.

Постановка Задачи: дан заранее обученный случайные лес. Необходимо найти области
пространства признаков в которых данный случайный лес возвращает свои
максимальное/минимальное значения.

\section{Существующие решения}

Так как функция задаваемая случайным лесом является трудно обратимой,
в настоящее время методы для её оптимизации не рассматривают внутреннюю
структуру случайного леса, а применяются общие методы для оптимизаций суррогатных
функции.

В данном случае основные подходы это:
\begin{enumerate}
    \item Случайный поиск --- выдирается случайная точка в пространстве,
    вычисляется значение функции в данной точке. Если полученный ответ лучше
    найденного ранее, то значение ответа обновляется.
    \item Локальный поиск --- поиск начинается из заданной точки, и на каждом шаге
    алгоритм переходит в соседнюю с лучшим значением оптимизируемой функции. Поиск
    останавливается при достижении локального экстремума или по истечении
    установленного времени.
\end{enumerate}

Также применяется комбинация этих двух алгоритмов, где производится локальный
поиск из набора случайных стартовых точек.

\section{Метод имитации отжига}

Популярным методом для решения задач оптимизации является метод имитации отжига.
Это общий алгоритмический метод решения задачи глобальной оптимизации, особенно
дискретной или комбинаторной. Основной идеей алгоритма, является имитация
физического процесса, происходящего при отжиге металлов, откуда алгоритм и берет
своё название.

Алгоритм производит случайные мутации по переходу из рассматриваемой точки
в соседнюю. При этом, переход в точку с худшим значением целевой функции
осуществляется с вероятностью, постепенно убывающей в соответствии с понижением
температуры:

\[
P(x_i \to x_{i+1}) =
\begin{cases}
    1,                                                  & F(x_i+1) > F(x_i) \\
    \exp \left(-\dfrac{F(x_{i+1})-F(x_i)}{T_i}\right),  & F(x_i+1) \geqslant F(x_i)
\end{cases}
\]

Где $P(x_i \to x_{i+1})$ --- вероятность события перехода в из точки $x_i$
в точку $x_{i+1}$. $F$ --- целевая функция. $T_i > 0$ --- убывающий
температурный параметр.

Основное преимущество данного алгоритма по сравнению с похожими методами
(локальный поиск) --- это случайность при переходе в следующее значение. Что
позволяет избегать остановки алгоритма в локальных экстремумах функции.

\section{Метод ветвей и границ}

Метод ветвей и границ (\texttt{branch and bound}) --- ещё один подход для
решения задач комбинаторной оптимизации. Основывается на методе полного
перебора, с отсечением тех вариантов, которые гарантированно не могут улучшить
найденное значение.

Основная идея метода заключается в том, что при некоторых ограничениях на
параметры целевой функции можно оценить её возможные значения. Тем самым можно
не перебирать те границы в которых гарантированно нет максимума функции, а те
границы в которых он потенциально может быть разбиваются на меньшие участки
и процесс повторяется.

Так как метод является вариацией полного перебора, то его эффективность сильно
зависит от качества оценки функции в заданных границах и правильного подбора
этих границ.

\chapterconclusion

В данной главе произведено теоретическое введение в модель машинного обучения
''случайный регрессионный лес''. Поставлена задача, решаемая в данной работе.
Также описаны и рассмотрены характеристики алгоритмов оптимизации, которые
используются на практике в данный монумент, и методы использованные в работе.

\chapter{Предложенные алгоритмы оптимизации случайного леса}\label{chap:second}

\section{Метод имитации отжига}\label{sec:odod}

Перед началом алгоритма все пространство разбивается на прямоугольную сетку, где
каждая прямая соответствует одному из порогов разветвлений в вершине одного из
деревьев. Для всех прямых также храниться информация в каком дереве происходит
разветвление по данной границе. Тем самым мы сводим задачу из вещественной
оптимизации к дискретному случаю, к чему более применим алгоритм имитации
отжига. Важно заметить, что по одному и тому же разбиению могут происходить
несколько вершин в разный деревьях или в одном дереве.

В процессе имитации отжига выполняются случайные мутации по переходу в соседнюю
клетку сетки. Так как при таком переходе пересекается лишь одна граница, для
получения нового значения достаточно пересчитать лишь те деревья в которых есть
эта граница, то есть те, что был и сохранены ранее для границ.

От температурного параметра зависит вероятность перехода в клетку не улучшающею
ответ найденный в текущей точке, формулы для расчёта вероятности описаны
в секции~\ref{sec:od}.

\subsection{Теоретическое сравнение}

\begin{itemize}
    \item \textbf{Преимущество метода:} Константное время работы, не
    зависящие от внутренней сложности устройства конкретных деревьев в лесе.
    В случае обновления леса новой информацией можно начинать поиск из точки
    минимума/максимума найденной ранее.
    \item \textbf{Недостатки метода:} Так как это общий метод оптимизации функции, то
    данный метод не использует внутреннею структуру случайного леса,
    а рассматривает его как 'чёрный ящик'. Кроме этого, часто переход в соседнюю
    клетку происходит по границе, которая не влияет на текущее значение,
    следовательно оно не изменяется. В итоге алгоритму приходится делать большое
    количество не существенных мутаций, что усложняет его работу.
\end{itemize}

\subsection{Результаты}

В результате экспериментов (Подробнее об это в секции~\ref{sec:error}) данный
метод не продемонстрировал желаемых результатов, и было принято решение
отказаться от него в пользу более эффективных. Возможно существуют более
эффективные реализации метода имитации отжига в применении к поставленной
задаче, не рассмотренные в данной работе. Возможное улучшение может быть, если
вместо того, чтобы на каждом шаге рассматривать все пороги всех вершин, можно
рассматривать только вершины на пути от текущей точки до корня дерева принятия
решений.

\section{Метод ветвей и границ}\label{sec:heu}

В алгоритме применяется полный перебор всех поддеревьев. В процессе перебора
поддерживается следующий инвариант: все рассматриваемые поддеревья имеют не
пустое пересечение. Тем самым в каждый момент времени рассматривается некоторая
область в виде n-мерного прямоугольника. На каждом шаге перебора, то есть
переходе из вершины в левого или правого ребёнка, эта область разрезается на две
части по границе из вершины, в которой был совершён шаг.

Тем самым для применения метода ветвей и границ необходимо оценить возможные
занижения возвращаемые лесом в рассматриваемом в конкретный момент времени
прямоугольнике. Для оценки максимума и минимума функции в этой области для
каждой вершины посчитано максимальное значение в её поддереве. Так как случайный
лес возвращает среднее значение всех деревьев, то в данной области лес не может
вернуть значение больше (меньше) чем среднее по всем достижимым максимумам
(минимумам) в поддеревьях, задающих эту область. Что позволяет нам получить
необходимые ограничения на оптимизируемую функцию.

В итоге на каждом шаге метода \emph{ветвей и границ} алгоритм сначала производит
\emph{ветвление} (усечение рассматриваемой области, путём спуска вниз на один
уровень в одном из деревьев принятия решения), после чего оценивает
\emph{границы} в текущей области, описанным раннее способом. Тем самым,
эффективность алгоритма зависит от выбора в каком из поддеревьев сначала
совершать обход, потому что, если алгоритм быстро найдёт верное, или же
достаточно близкое значение, то тогда он сможет пропустить большое количество
гарантированно неэффективных поддеревьев.

Далее описана эвристика в процессе выбора следующего поддерева, которая как,
в последствии будет повреждено экспериментально, значительно ускоряет процесс
обхода случайного леса.

\subsection{Эвристика}

Основная эвристическая оптимизация --- перебирать сначала те поддеревья,
в которых разница между оценёнными значениями левого и правого поддерева
максимальна.

\[
    i = \arg \max_{v \in trees}(|value[v.left] - value[v.right]|)
\]

Идея основывается на предположении, что после обхода поддерева первого ребёнка
в такой паре. Из-за максимальной разницы между ними следует большая вероятность
того, что алгоритму не придётся рассматривать поддерево второго ребёнка, так как
в первом поддереве было найдено лучшее значение.

\subsection{Нахождение приближенного значения}

Для поиска ответа с заданной точностью применена следующая оптимизация. Алгоритм
не рассматривает те поддеревья которые гарантированно не превосходят ранее
найденный ответ на больше чем заданное $\alpha$.

\[
    value[v] < \alpha current
\]

Это позволяет находить ответ отличающийся от истинного не более чем на заданную
точность, потому что если найденный ответ отличается больше, то не было
рассмотрено его поддерево, что невозможно так как возможный максимум в нем
больше.

Отметим, что с линейным увеличением допустимой погрешности $\alpha$ алгоритм
раньше (на меньшей глубине дерева) прекратит обход неэффективного поддерева.
А так как количество вершин в поддереве растёт экспоненциально в зависимости от
его глубины, то таким образом с линейным ростом погрешности экспоненциально
растёт количество пропущенных вершин, а следовательно уменьшается время работы.
Это будет подтверждено экспериментально далее в работе.

\subsection{Уточнение границ в поддереве}

Заметим, что максимум в поддереве может не пересекаться с областью, которую
рассматривает алгоритм в конкретный момент. Из-за этого алгоритм может
перебирать поддеревья, максимум в которых заведомо не достижим.

Чтобы это исправить для каждого дерева принятия решений посчитан отсортированный
список всех его листьев. Данный список строиться путём последовательного слияния
списков детей в каждой вершине, что асимптотически добавляет $O(N \log{N})$
времени к предподсчету алгоритма, где $N$ --- количество листьев в дереве.

Так как на каждом шаге алгоритма новая полученная область строго включается
в старую, в таком отсортированном списке возможный максимум в поддереве строго
движется вперёд. В итоге за добавление линейного времени на каждый спуск
возможно оценить максимум в поддереве, пересекающийся с текущей рассматриваемой
областью.

Следует заметить, что данную оптимизацию следует применять лишь в случае, когда
в исходном наборе данных количество признаков не слишком велико (примерно
$<20$\footnote{эмпирически полученное значение в результате серии
экспериментов}), так как на каждом шаге приходиться вычислять пересечение
текущей рассматриваемой области, с областями соответствующими листьям деревьев.
А данное пересечение выполняется за $O(m)$, где $m$ --- количество признаков
в данных. Также улучшение заметно лишь при большем количестве деревьев
в случайном лесу.

\subsection{Оптимизация в заданной области}

Отметим, что в данной реализации алгоритма, можно задать начальные ограничения
на пространство признаков, в котором происходит поиск максимума или минимума.
Для этого достаточно задать начальную область алгоритма не все пространство,а
требуемую область. Так как алгоритм в своей работе поддерживает, что все
поддеревья пересекаются с текущей рассматриваемой областью, алгоритм
автоматически будет рассматривать, только поддеревья в заданной границе. Что
позволяет решать задачу оптимизации не на всем пространстве, а в конкретной
области.

\section{Практическое применение}\label{sec:smac}

Описанный выше алгоритм может быть приманен для улучшения существующих
алгоритмов. Поэтому, чтобы продемонстрировать применимость разработанного метода
на практике, в данном исследовании произведена успешная попытка улучшить
алгоритм подбора гиперпараметров SMAC, применением в нем поиска минимума
в случном лесе с помощь. предложенного алгоритма.

\subsection{Лес случайных деревьев в алгоритме SMAC}

В ходе своей работы SMAC использует лес случайных деревьев для предсказания
ожидаемого улучшения при использовании соответствующего набора гиперпараметров.

Алгоритм состоит из нескольких итераций. На каждой итерации, сначала
регрессионная модель (в данном случае случайны лес) обновляется новыми данными,
после чего собираются новые данные в точках выбранных на основе обновленной
модели.

\subsection{Модификация с применением разработанного алгоритма}

Были модифицированы \texttt{automl} реализации с открытым исходным кодом:
\texttt{random\_forest\_run} и \texttt{SMAC}. В \texttt{random\_forest\_run} был
добавлен эвристический алгоритм для поиска минимальной и максимальной области.
Последняя оптимизация не была реализована, так как она рассчитана на большое
количество деревьев. А в \texttt{automl} реализации SMAC используется случайный
лес с $10$ деревьями, и в таком случае данная оптимизация лишь замедляет работу
алгоритма.

В предложенной реализации алгоритма SMAC случайный выбор конфигураций
гиперпараметров заменён на поиск минимального значения в текущей регрессионной
можели (случайный лес). Алгоритм возвращает $\gamma * size$ различных
конфигураций из найденной максимальной области случайного леса, и оставшиеся $(1
- \gamma) * size$ выбираются случайно. Это делается потому, что текущий
регрессионной лес может не точно отражать, реальное поведение целевого
алгоритма. Тем самым алгоритм избегает переобучения регрессионной модели.
Параметр $\gamma$ может быть задан пользователем в аргументах оптимизатора,
исходя из конкретного применения предложенного алгоритма. В данной работе
применяется $\gamma=0.8$ выбранное эмпирически, как более эффективное
в большинстве случаев.

\chapterconclusion

В данной главе предложены два метода оптимизации случайного регрессионного леса
применённые в работе, и как эти методы применяются для решения задачи,
поставленной в работе. Изложены применённые оптимизации к этим методам и детали
реализации в конкретной работе.

Отметим, что если метод имитации отжига является идейным продолжении
применяющихся сейчас на практике решений поставленной задачи, то метод ветвей
и границ является кардинально новым подходом к решению оптимизации случайного
леса.

При экспериментальном исследовании, от метода имитации отжига было принято решение
отказаться в пользу метода ветвей и границ.

Также в данной главе описано практическое применения разработанного алгоритма
поиска минимума в случайном лесе. Предложена модификация алгоритма выбора
конфигураций гиперпараметров SMAC\@.

\chapter{Тестирование}

\section{Реализация алгоритма}

За основу была взята реализация алгоритма случайного регрессионного леса
\texttt{RandomForestRegressor} из \texttt{Python} библиотеки
\texttt{sklearn-kit}. 

К стандартному функционалу леса добавлена функция вычисления максимума или
минимума на всем пространстве или же заранее заданном его подпространстве,
с возможностью выбора алгоритма оптимизации (случайный поиск, метод ветвей
и границ, метод ветвей и границ с применением эвристик), допустимой погрешности
(для алгоритмов на основе ветвей и границ) и количества итераций (для алгоритмов
на основе случайного или локального поиска). 

\section{Параметры и метод тестирования}

Для обучения случайных лесов применялись общедоступные датасеты из общедоступен
базы данных OpenML, их параметры приведены в таблице~\ref{tab_datasets}.

\begin{center}
    \begin{table}[!ht]
    \caption{Таблица с параметрами использованных датасетов}\label{tab_datasets}
        \begin{tabular}{|l|l|l|}

\hline

название        & элементы  & признаки \\

\hline

diabetes        & 442    & 9     \\
boston          & 506    & 12    \\
autoPrice       & 159    & 16    \\
wisconsin       & 194    & 33    \\
strikes         & 625    & 7     \\
kin8nm          & 8192   & 9     \\
house\_8L       & 22784  & 9     \\
house\_16H      & 22784  & 9     \\
mtp2            & 274    & 1143  \\

\hline

\end{tabular}

    \end{table}
\end{center}

В работе было произведено сравнение следующих алгоритмов следующие алгоритмы: 

\begin{enumerate}
        \item Перебор
        \item Перебор с погрешностью $<5\%$
        \item Случайный
        \item Отжиг
        \item Эвристика
        \item Эвристика с погрешностью $<5\%$
        \item Эвристика с погрешностью $<15\%$
\end{enumerate}

На каждом наборе параметров (то есть: датасет, количество деревьев, максимальная
глубина) обучалось 10 лесов с разными случайными начальными значениями, после
чего на каждом запускались все рассматриваемые алгоритмы оптимизации. По
результатам запусков вычислялось математическое ожидание и среднеквадратическое
отклонение времени работы алгоритма. Так же вычислялась относительная
погрешность результата по следующей формуле:

\[
    \sigma = \frac{Max_{correct} - Max_{found} + Min_{correct} - Min_{found}}
    {Max_{correct} - Min_{correct}}
\]

\section{Результаты}

\begin{center}

\begin{figure}[!ht]
    \caption{Сравнение времени работы}\label{time}
    \begin{tabular}{c c}
        \timeeasy{0.45\textwidth}{0.35\textheight} &
        \timetrees{0.45\textwidth}{0.35\textheight} \\
    \end{tabular}
\end{figure}

\begin{figure}[!ht]
    \caption{Сравнение погрешности работы}\label{error}
    \begin{tabular}{c c}
        \erroreasy{0.45\textwidth}{0.35\textheight} &
        \errorbig{0.45\textwidth}{0.35\textheight}\\
    \end{tabular}
\end{figure}

\begin{figure}[!ht]
    \caption{Сравнение времени работы в зависимости от погрешности}\label{error_to_time}
    \errortotime{1.0\textwidth}{0.6\textheight}
\end{figure}

\end{center}

\subsection{Время работы}

На графиках (Рисунок~\ref{time}) изображено сражение рассматриваемых
алгоритмов Как видно из графиков время работы эвристического алгоритма
оказывается меньше других алгоритмов, особенно при допуске большой
погрешности.

\subsection{Точность}

Как видно из графиков (Рисунок~\ref{error}), даже при большой погрешности эвристический алгоритм
оказывается более точным нежели другие алгоритмы в сравнении.

\chapterconclusion{}

Эвристика демонстрирует лучшее время работы и при даже большой заданной
погрешности находит результат более точный чем остальные алгоритмы.

\chapter{Практическое применение}

\section{SMBO и SMAC}

Последовательная оптимизация основанная на модели (SMBO) Sequential Model-based
Algorithm configuration\cite{smbo} (SMAC)

\begin{enumerate}
\item Использует случайный лес в качестве регрессионной модели\cite{usesmbo}
\item В лес добавляются текущие результаты алгоритмов
\item По значениям леса выбираются новые конфигурации алгоритма
\end{enumerate}

\section{Модификация с применением разработанного алгоритма}
    
Были модифицированы \texttt{automl} реализации с открытым исходным кодом:
\texttt{random\_forest\_run} и \texttt{SMAC}. В \texttt{random\_forest\_run} был
добавлен эвристический алгоритм для поиска минимальной и максимальной области.
Последняя оптимизация не была реализована, так как она рассчитана на большое
количество деревьев. А в \texttt{automl} реализации SMAC используется случайный
лес с 10 деревьями, и в таком случае данная оптимизация лишь замедляет работу
алгоритма.

В модифицированной версии SMAC случайный выбор конфигураций гиперпараметров,
возвращение $\alpha * size$ конфигураций из найденной максимальной области
случайного леса, и оставшиеся $(1 - \alpha) * size$ выбираются случайно. Это
делается потому, что текущий регрессионной лес может не точно отражать, реальное
поведение целевого алгоритма.

\section{Тестирование и сравнение}

Для тестирования подбирались гиперпараметры для дерева принятия решений. Этот
выбран исходя из следующих соображений:

\begin{itemize}
    \item Быстрый и лёгкий алгоритм, не требующий сложных и длительный вычислений.
    Так как целевой алгоритм запускается порядка 1000 раз за один тест.
    \item Алгоритм требует подбора гиперпараметров, и без этого работает крайне не
    эффективно. (оценка точности 0.2 с стандартной конфигурацией)
\end{itemize}

На графике (Рисунок~\ref{smac_size}) изображена зависимость оценки точности
целевого алгоритма от размера пространства гиперпараметров. То есть от
количества различных комбинаций гиперпараметров. Для этого изначальный набор
гиперпараметров был искусственно ограничен, на соответствующий коэффициент по
всем параметрам.

На графике (Рисунок~\ref{smac_count}) изображена зависимость оценки точности
целевого алгоритма от ограничения на количество запусков целевого алгоритма.

\begin{center}
\begin{figure}[!ht]
\caption{Зависимость от размера пространства гиперпараметров}\label{smac_size}
\smacsize{0.8\textwidth}{0.4\textheight}
\end{figure}
\begin{figure}[!ht]
\caption{Зависимость от размера пространства гиперпараметров}\label{smac_count}
\smaccount{0.8\textwidth}{0.4\textheight}
\end{figure}
\end{center}

\chapterconclusion{}

Разработанный алгоритм находит своё применение в работе алгоритма подбора
гиперпараметров SMAC, показывая хорошие результаты и большую устойчивость
к размеру пространства гиперпараметров.


%% Макрос для заключения. Совместим со старым стилевиком.
\startconclusionpage{}
В данной работе был предложен новый подход к решению задачи оптимизации функции
заданной случайным лесом.

Было проведено масштабное сравнение предложенных методов и существующих сейчас
алгоритмов. В результате которого метод ветвей и границ был принят как самый
эффективный. Так же проведено сравнение его работы при различных настраиваемых
параметрах и в приложении на по разному обученные случайные леса.

После чего этот метод был реализован в применении на практической задачи выбора
модели и настройки её гиперпараметров. Предложенный алгоритм продемонстрировал
статистически значимые лучшие результаты. Он может успешно применяться в любых
случаях оптимизации леса.

Таким образом, цель достигнута и все задачи выполнены.

\printmainbibliography%

\end{document}
