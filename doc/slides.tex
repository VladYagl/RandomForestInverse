\documentclass{beamer}

%Russian-specific packages
%--------------------------------------
\usepackage[T2A]{fontenc}
\usepackage[utf8]{inputenc}
\usepackage[english,main=russian]{babel}
%--------------------------------------

\usepackage{graphicx}
\usepackage{amssymb,amsmath}
\usepackage{pgfplots, pgfplotstable}
\pgfplotsset{compat=newest}
\usepackage{pgfplots, pgfplotstable}
\usepgfplotslibrary{fillbetween}
\pgfplotsset{compat=newest}

\makeatletter
\pgfplotsset{%
    /pgfplots/flexible xticklabels from table/.code n args={3}{%
        \pgfplotstableread[#3]{#1}\coordinate@table
        \pgfplotstablegetcolumn{#2}\of{\coordinate@table}\to\pgfplots@xticklabels
        \let\pgfplots@xticklabel=\pgfplots@user@ticklabel@list@x
    }
}
\makeatother%
\definecolor{basicno}{rgb}{0.624,0.0,0.0}
\definecolor{basicsome}{rgb}{0.424,0.0,0.0}
\definecolor{random}{rgb}{0.0,0.100,0.424}
\definecolor{gena}{rgb}{0.0,0.100,0.624}
\definecolor{heuno}{rgb}{0.2,0.524,0.2}
\definecolor{heusome}{rgb}{0.2,0.424,0.2}
\definecolor{heumore}{rgb}{0.1,0.324,0.1}

\newcommand{\graphwidth}{\textwidth}
\newcommand{\graphheight}{0.8\textheight}

\graphicspath{{logo/}{pics/}}
\beamertemplatenavigationsymbolsempty{}
\setbeamertemplate{footline}[frame number]
\setbeamerfont{footline}{size={\fontsize{10}{12}}}

\title{Оптимизация функции, задаваемой регрессионным лесом}
\author{Влад Ягламунов}
\date{}

\begin{document}

\begin{frame}
\thispagestyle{empty}
\maketitle
{\small 
    \textbf{Формальный научный руководитель:}\par Фильченков Андрей Александрович\par
    \textbf{Фактический научный руководитель:}\par Шаламов Вячеслав Владимирович
}
\end{frame}

\begin{frame} \frametitle{Задача}
    \begin{itemize}
        \item \textbf{Дано:} Обученный регрессионный лес
        \item \textbf{Найти:} Области, где лес возвращает минимальное и максимальное значение
    \end{itemize}
\end{frame}

\begin{frame} \frametitle{Применение}
    Random Forest:
    \begin{itemize}
        \item Суррогатные функции для оптимизации
        \item Хорошо обновляется при добавлении информации
        \item Сложно обратимая функция \pause{}
        \item Последовательная оптимизация основанная на модели Sequential Model-Based Optimization (SMBO)
        \item Sequential Model-based Algorithm configuration (SMAC)
    \end{itemize}
    \vfill
    \pause{}
    Сейчас используется перебор случайного набора точек.
    \vfill
\end{frame}

\begin{frame} \frametitle{Случайный лес}
    \begin{columns}
        \column{.6\textwidth}
            \begin{itemize}
                \item Ансамбль деревьев принятия решения
                \item Каждое дерево обучено на случной подвыборке
                \item Результат: среднее всех деревьев 
                \item Пространство разбивается на прямоугольники по границам разветвления вершин
            \end{itemize}
        \column{.4\textwidth}
        \includegraphics[width=\textwidth]{random_forest.png}
    \end{columns}
\end{frame}

\begin{frame} \frametitle{Алгоритм имитации отжига}
    \begin{center}
    \includegraphics[width=0.8\textwidth]{gena.png}
    \end{center}
    \only<1>{%
        \begin{itemize}
            \item Разбиваем пространство по всем границам
            \item Случайные мутации по переходе в соседнюю клетку
            \item Достаточно пересчитать только одно дерево
            \item Если значение ухудшилось, то переходим с вероятностью уменьшающейся от температурного параметра
        \end{itemize}
    }
    \only<2>{%
        \begin{itemize}
            \item Рассматривает лес как 'чёрный ящик', не использует внутреннею структуру случайного леса
            \item Не работает при большом количестве признаков (>1000)
            \item Не показал желаемых результатов
        \end{itemize}
    }
\end{frame}

\begin{frame} \frametitle{Эвристический алгоритм}
    \begin{columns}
        \column{.5\textwidth}
            \begin{itemize}
                \item Перебор по всем поддеревьям
                \item Для вершины храним минимум и максимум в поддереве
                \item Поддерживаем, что все поддеревья пересекаются
            \end{itemize}
        \column{.5\textwidth}
            \begin{center}
            \includegraphics[height=0.8\textheight]{tree.png}

            алгоритм рассматривает синие вершины двух деревьев
            \end{center}
    \end{columns}
\end{frame}

\begin{frame} \frametitle{Эвристический алгоритм}

    \begin{columns}
        \column{.75\textwidth}
            На каждом шаге рассматриваем n-мерный прямоугольник области

            Шаг: разбиение по границе вершины
        \column{.25\textwidth}
            \includegraphics[width=\textwidth]{split.png}
    \end{columns}
    \vfill
    \pause{}
    Эвристика:
    \[
        i = \arg \max_{v \in trees}(|value[v.left] - value[v.right]|)
    \]
\end{frame}

\begin{frame} \frametitle{Эвристический алгоритм (Оптимизация 1)}
    Необязательно искать точное решение.

    \vspace{50px}
    Не будем перебирать поддерево, если в лучшем случае это не улучшит ответ хотя бы на $\alpha$

    \[
        value[v] < \alpha current
    \]

    Гарантирует погрешность $<\alpha$
\end{frame}

\begin{frame} \frametitle{Эвристический алгоритм (Оптимизация 2)}
    Максимум в поддереве может не пересекаться с текущей областью

    \begin{columns}
        \column{.5\textwidth}
            \begin{itemize}
                \item Отсортированный список всех его листьев
                \item На каждом шаге в таком списке мы движемся только вперёд
            \end{itemize}
        \column{.5\textwidth}
            \includegraphics[width=1.1\textwidth]{merge.png}
    \end{columns}
\end{frame}

\begin{frame} \frametitle{Сравниваемые алгоритмы}
    \begin{itemize}
        \item Перебор
        \item Перебор с погрешностью $<5\%$
        \item Случайный
        \item Эвристика
        \item Эвристика с погрешностью $<5\%$
        \item Эвристика с погрешностью $<15\%$
    \end{itemize}
\end{frame}

\begin{frame} \frametitle{Использованные данные}
    Использованные различные общедоступные датасеты с OpenML

    \vfill
    \begin{center}
        \begin{tabular}{|l|l|l|}

\hline

название        & элементы  & признаки \\

\hline

diabetes        & 442    & 9     \\
boston          & 506    & 12    \\
autoPrice       & 159    & 16    \\
wisconsin       & 194    & 33    \\
strikes         & 625    & 7     \\
kin8nm          & 8192   & 9     \\
house\_8L       & 22784  & 9     \\
house\_16H      & 22784  & 9     \\
mtp2            & 274    & 1143  \\

\hline

\end{tabular}

        \vfill
        На каждом наборе параметров проводилось 10 тестов.
    \end{center}
\end{frame}

\begin{frame} \frametitle{Тестирование (Время работы | Простой случай)}
    \begin{center}
        \input{./include/time_easy.tex}
    \end{center}
\end{frame}

\begin{frame} \frametitle{Тестирование (Время работы | Много деревьев)}
    \begin{center}
        \input{./include/time_trees.tex}
    \end{center}
\end{frame}

\begin{frame} \frametitle{Тестирование (Время работы | Большой датасет)}
    \begin{center}
        \input{./include/time_big.tex}
    \end{center}
\end{frame}

\begin{frame} \frametitle{Тестирование (Погрешность | Простой случай)}
    \begin{center}
        \input{./include/error_easy.tex}
    \end{center}
\end{frame}

\begin{frame} \frametitle{Тестирование (Погрешность | Много признаков)}
    \begin{center}
        \input{./include/error_big.tex}
    \end{center}
\end{frame}

\begin{frame} \frametitle{Практическое применение | Подбор гиперпараметров алгоритмов}
    \only<1>{%
        Последовательная оптимизация основанная на модели (SMBO) Sequential Model-based Algorithm configuration (SMAC)
        \begin{enumerate}
        \item Использует случайный лес в качестве регрессионной модели
        \item В лес добавляются текущие результаты алгоритмов
        \item По значениям леса выбираются новые конфигурации алгоритма
        \end{enumerate}
        Были модифицированы \texttt{automl} реализации с открытым исходным кодом: \texttt{random\_forest\_run} и \texttt{SMAC}
    }
\end{frame}

\begin{frame} \frametitle{Тестирование | SMAC}
    \begin{center}
        \newcommand{\smacgeneral}{%
\pgfplotstableread[col sep=comma, header=false]{../data/smac/forest.csv}\forest
\pgfplotstableread[col sep=comma, header=false]{../data/smac/random.csv}\random

\pgfplotstablecreatecol[copy column from table={\random}{[index] 2}] {4} {\forest}
\pgfplotstablecreatecol[copy column from table={\random}{[index] 3}] {5} {\forest}

\pgfplotstableset{%
    bold/.style = {%
        postproc cell content/.style={%
            @cell content/.add={\boldmath}{},
        },
    },
}

\pgfplotstabletypeset[
    every head row/.style={%
        output empty row,
        before row={%
            \hline
            \multicolumn{1}{|c}{Алгоритм} & \multicolumn{1}{|c}{Набор данных} &
            \multicolumn{3}{|c}{Модификация} & \multicolumn{3}{|c|}{Оригинал} \\
        },
    },
    before row=\hline,
    every last row/.style={after row=\hline},
    columns={0, 1, forestacc, plusminus, 3, randomacc, plusminus, 5},
    display columns/0/.style= {%
            column name=Алгоритм,
            string type,
            column type = {|c|},
    },
    create on use/plusminus/.style={create col/set={$\pm$}},
    columns/plusminus/.style= {column name=, string type, column type = {b{0mm}}},
    display columns/1/.style= {%
            column name=Выборка,
            string type,
            column type = {c|},
    },
    create on use/forestacc/.style={%
        create col/expr={1-\thisrow{2}}
    },
    columns/forestacc/.style= {%
            column name=,
            fixed,
            precision=2,
            column type = {b{6mm}},
    },
    columns/3/.style= {%
            column name=,
            fixed,
            precision=2,
            column type = {l|},
    },
    create on use/randomacc/.style={%
        create col/expr={1-\thisrow{4}}
    },
    columns/randomacc/.style= {%
            column name=,
            fixed,
            precision=2,
            column type = {b{6mm}},
    },
    columns/5/.style= {%
            column name=,
            fixed,
            precision=2,
            column type = {l|},
    },
    every row 0 column 2/.style = {bold},
    every row 3 column 2/.style = {bold},
    every row 4 column 2/.style = {bold},
    every row 5 column 2/.style = {bold},
    every row 8 column 2/.style = {bold},
    every row 9 column 2/.style = {bold},
    every row 13 column 2/.style = {bold},
    every row 0 column 3/.style = {bold},
    every row 3 column 3/.style = {bold},
    every row 4 column 3/.style = {bold},
    every row 5 column 3/.style = {bold},
    every row 8 column 3/.style = {bold},
    every row 9 column 3/.style = {bold},
    every row 13 column 3/.style = {bold},
    every row 0 column 4/.style = {bold},
    every row 3 column 4/.style = {bold},
    every row 4 column 4/.style = {bold},
    every row 5 column 4/.style = {bold},
    every row 8 column 4/.style = {bold},
    every row 9 column 4/.style = {bold},
    every row 13 column 4/.style = {bold},
]{\forest}
}

    \end{center}
\end{frame}

\begin{frame} \frametitle{Тестирование | SMAC, размер пространства}
    \begin{center}
        \input{./include/smac_size.tex}
    \end{center}
\end{frame}

\begin{frame} \frametitle{Тестирование | SMAC, количество запусков алгоритма}
    \begin{center}
        \input{./include/smac_count.tex}
    \end{center}
\end{frame}

\begin{frame} \frametitle{Итоги}
    \begin{itemize}
        \item Разработан алгоритм оптимизации случайного леса
        \item Проведено его сравнение с существующими методами
        \item Проверено его практическое применение в сочетании с другими алгоритмами
    \end{itemize}
\end{frame}

\end{document}
