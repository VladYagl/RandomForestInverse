\chapter{Предложенные алгоритмы оптимизации случайного леса}

\section{Метод имитации отжига}

Все пространство разбивается на прямоугольную сетку, где каждая прямая
соответствует одной из границ разветвлений в вершине одного из деревьев. Для
всех прямых также храниться информация в каком дереве происходит разветвление по
данной границе. Важно заметить, что по одному и тому же разбиению могут
происходить несколько вершин в разный деревьях или в одном дереве.

В процессе имитации отжига выполняются случайные мутации по переходу в соседнюю клетку сетки.
Так как при таком переходе пересекается лишь одна граница, для получения нового
значения достаточно пересчитать лишь те деревья в которых есть эта граница, то
есть те, что был и сохранены ранее для границ.

\subsection{Теоретическое сравнение}

\begin{itemize}
    \item \textbf{Преимущество метода:} Константное время работы, не
    зависящие от внутренней сложности устройства конкретных деревьев в лесе.
    \item \textbf{Недостатки метода:} Так как это общий метод оптимизации функции, то
    данный метод не использует внутреннею структуру случайного леса,
    а рассматривает его как 'чёрный ящик'. Кроме этого, часто переход в соседнюю
    клетку происходит по границе, которая не влияет на текущее значение,
    следовательно оно не изменяется. В итоге алгоритму приходится делать большое
    количество не существенный мутаций, что усложняет его работу.
\end{itemize}

\subsection{Результаты}
На практике данный метод не показал желаемых результатов, и было принято решение
отказаться от него в пользу более эффективных. Возможно существуют более эффективные
реализации метода имитации отжига, не рассмотренные в данной работе.


\section{Метод ветвей и границ}

В алгоритме применяется полный перебор всех поддеревьев. В процессе перебора
поддерживается следующий инвариант: все рассматриваемые поддеревья имеют не
пустое пересечение. Тем самым в каждый момент времени рассматривается некоторая
область в виде n-мерного прямоугольника. На каждом шаге перебора, то есть
переходе из вершины в левого или правого ребёнка, эта область разрезается на две
части по границе из вершины, в которой был совершён шаг.

Тем самым границей на функцию является рассматриваемый в конкретный момент
времени прямоугольник. Для оценки максимума и минимума функции в этой области
для каждой вершины посчитано максимальное значение в её поддереве. Так как
случайный лес возвращает среднее значение всех деревьев, то в данной области лес
не может вернуть значение больше (меньше) чем среднее по всем достижимым
максимумам (минимумам) в поддеревьях, задающих эту область. Что позволяет нам
получить необходимые ограничения на оптимизируемую функцию.

В итоге алгоритм не будет рассматривать поддеревья которые гарантированно не
превосходят ранее найденный ответ.

\subsection{Эвристика}

Основная эвристическая оптимизация --- перебирать сначала те поддеревья,
в которых разница между левым и правым детьми максимальна.

\[
    i = \arg \max_{v \in trees}(|value[v.left] - value[v.right]|)
\]

Идея основывается на предположении, что после обхода поддерева первого ребёнка
в такой паре. Из-за максимальной разницы между ними следует большая вероятность
того, что алгоритму не придётся рассматривать поддерево второго ребёнка.

\subsection{Нахождение приближенного значения}

Для поиска ответа с заданной точностью применена следующая оптимизация. Алгоритм
не рассматривает те поддеревья которые гарантированно не превосходят ранее
найденный ответ на больше чем заданное $\alpha$.

\[
    value[v] < \alpha current
\]

Это позволяет находить ответ отличающийся от истинного не более чем на заданную
точность, потому что если найденный ответ отличается больше, то не было
рассмотрено его поддерево, что невозможно так как возможный максимум в нем
больше.

\subsection{Уточнение границ в поддереве}

Заметим, что максимум в поддереве может не пересекаться с областью, которую
рассматривает алгоритм в конкретный момент. Из-за этого алгоритм может
перебирать поддеревья, максимум в которых заведомо не достижим.

Чтобы это исправить для каждого дерева принятия решений посчитан отсортированный
список всех его листьев. Данный список строиться путём последовательного слияния
списков детей в каждой вершине, что асимптотически добавляет $O(N \log{N})$
времени к предподсчету алгоритма, где $N$ --- количество листьев в дереве.

Так как на каждом шаге алгоритма новая полученная область строго включается
в старую, в таком отсортированном списке возможный максимум в поддереве строго
движется вперёд. В итоге за добавление линейного времени на каждый спуск
возможно оценить максимум в поддереве, пересекающийся с текущей рассматриваемой
областью.

\subsection{Оптимизация в заданной области}

Отметим, что в данной реализации алгоритма, можно задать начальные ограничения
на пространство признаков, в котором происходит поиск максимума или минимума.
Что позволяет решать задачу оптимизации не на всем пространстве, а в конкретной
области.

\chapterconclusion

В данной главе предложены два метода оптимизации случайного регрессионного леса
применённые в работе, и как эти методы применяются для решения задачи,
поставленной в работе. Изложены применённые оптимизации к этим методам и детали
реализации в конкретной работе.

Если метод отжига является идейным продолжении применяющихся сейчас на практике
решений поставленной задачи, то метод ветвей и границ является кардинально новым
подходом к решению оптимизации случайного леса.

При экспериментальном исследовании, от метода имитации отжига было принято решение
отказаться в пользу метода ветвей и границ.
